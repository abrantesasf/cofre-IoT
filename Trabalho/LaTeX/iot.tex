%%%%%%%%%%%%%%%%%%%%%%%%%%%%%%%%%%%%%%%%%%%%%%%%%%%%%%%%%%%%%%%%%%%%%%%%%%%%%%%%%
% Template: Article
%
% Por: Abrantes Araújo Silva Filho
%      abrantesasf@gmail.com
%
% Citação: Se você gostou deste template, por favor ajude a divulgá-lo mantendo
%          o link para meu repositório GitHub em:
%          https://github.com/abrantesasf/LaTeX
%%%%%%%%%%%%%%%%%%%%%%%%%%%%%%%%%%%%%%%%%%%%%%%%%%%%%%%%%%%%%%%%%%%%%%%%%%%%%%%%%





%%%%%%%%%%%%%%%%%%%%%%%%%%%%%%%%%%%%%%%%%%%%%%%%%%%%%%%%%%%%%%%%%%%%%%%%%%%%%%%%%
%%% Configura o tipo de documento, papel, tamanho da fonte e informações básicas
%%% para as proriedades do PDF/DVIPS e outras propriedades do documento
\RequirePackage{ifpdf}
\ifpdf
  % Classe, língua e tamanho da fonte padrão. Outras opções a considerar:
  %   draft
  %   onecolumn (padrão) ou twocolumn (OU usar o package multicol)
  %   fleqn com ou sem leqno (alinhamento à esquerda das fórmulas e dos números)
  %   oneside (padrão para article ou report) ou twoside (padrão para book)
  \documentclass[pdftex, brazil, 12pt, twoside]{article}
\else
  % Classe, língua e tamanho da fonte padrão. Outras opções a considerar:
  %   draft
  %   onecolumn (padrão) ou twocolumn (OU usar o package multicol)
  %   fleqn com ou sem leqno (alinhamento à esquerda das fórmulas e dos números)
  %   oneside (padrão para article ou report) ou twoside (padrão para book)
  \documentclass[brazil, 12pt]{article}
\fi


%%%%%%%%%%%%%%%%%%%%%%%%%%%%%%%%%%%%%%%%%%%%%%%%%%%%%%%%%%%%%%%%%%%%%%%%%%%%%%%%%
%%% Carrega pacotes iniciais necessários para estrutura de controle e para a
%%% criação e o parse de novos comandos
\usepackage{ifthen}
\usepackage{xparse}


%%%%%%%%%%%%%%%%%%%%%%%%%%%%%%%%%%%%%%%%%%%%%%%%%%%%%%%%%%%%%%%%%%%%%%%%%%%%%%%%%
%%% Configuração do tamanho da página, margens, espaçamento entrelinhas e, se
%%% necessário, ativa a indentação dos primeiros parágrafos.
\ifpdf
  \usepackage[pdftex]{geometry}
\else
  \usepackage[dvips]{geometry}
\fi
\geometry{a4paper, left=2.6cm, right=4.0cm, top=3.0cm, bottom=3.4cm}

\usepackage{setspace}
  %\singlespacing
  \onehalfspacing
  %\doublespacing


%%%%%%%%%%%%%%%%%%%%%%%%%%%%%%%%%%%%%%%%%%%%%%%%%%%%%%%%%%%%%%%%%%%%%%%%%%%%%%%%%
%%% Configurações de cabeçalho e rodapé:
\usepackage{fancyhdr}
\setlength{\headheight}{1cm}
\setlength{\footskip}{1.5cm}
\renewcommand{\headrulewidth}{0.3pt}
\renewcommand{\footrulewidth}{0.0pt}
\pagestyle{fancy}
\renewcommand{\sectionmark}[1]{%
  \markboth{\uppercase{#1}}{}}
\renewcommand{\subsectionmark}[1]{%
  \markright{\uppercase{\thesubsection \hspace{0.1cm} #1}}{}}
\fancyhead{}
\fancyfoot{}
\newcommand{\diminuifonte}{%
    \fontsize{9pt}{9}\selectfont
}
\newcommand{\aumentafonte}{%
    \fontsize{12}{12}\selectfont
}
% Configura cabeçalho e rodapé para documentos TWOSIDE
\fancyhead[EL]{\textbf{\thepage}}
\fancyhead[EC]{}
\fancyhead[ER]{\diminuifonte \textbf{\leftmark}}
\fancyhead[OR]{\textbf{\thepage}}
\fancyhead[OC]{}
\fancyhead[OL]{\diminuifonte \textbf{\rightmark}}
\fancyfoot[EL,EC,ER,OR,OC,OL]{}
% Configura cabeçalho e rodapé para documentos ONESIDE
%\lhead{ \fancyplain{}{sup esquerdo} }
%\chead{ \fancyplain{}{sup centro} }
%\rhead{ \fancyplain{}{\thesection} }
%\lfoot{ \fancyplain{}{inf esquerdo} }
%\cfoot{ \fancyplain{}{inf centro} }
%\rfoot{ \fancyplain{}{\thepage} }




%%%%%%%%%%%%%%%%%%%%%%%%%%%%%%%%%%%%%%%%%%%%%%%%%%%%%%%%%%%%%%%%%%%%%%%%%%%%%%%%%
%%% Configurações de encoding, lingua e fontes:
\usepackage[T1]{fontenc}
\usepackage[utf8]{inputenc}
\usepackage{babel}

% Altera a fonte padrão do documento (nem todas funcionam em modo math):
%   phv = Helvetica
%   ptm = Times
%   ppl = Palatino
%   pbk = bookman
%   pag = AdobeAvantGarde
%   pnc = Adobe NewCenturySchoolBook
\renewcommand{\familydefault}{ppl}


%%%%%%%%%%%%%%%%%%%%%%%%%%%%%%%%%%%%%%%%%%%%%%%%%%%%%%%%%%%%%%%%%%%%%%%%%%%%%%%%%
%%% Carrega pacotes para referências cruzadas, citações dentro do documento,
%%% links para internet e outros.Configura algumas opções.
%%% Não altere a ordem de carregamento dos packages.
\usepackage{varioref}
\ifpdf
  \usepackage[pdftex]{hyperref}
    \hypersetup{
      % Informações variáveis em cada documento (MUDE AQUI!):
      pdftitle={Internet das Coisas},
      pdfauthor={Abrantes Araújo Silva Filho},
      pdfsubject={Internet das Coisas},
      pdfkeywords={internet das coisas, internet of things, iot},
      pdfinfo={
        CreationDate={}, % Ex.: D:AAAAMMDDHH24MISS
        ModDate={}       % Ex.: D:AAAAMMDDHH24MISS
      },
      % Coisas que você não deve alterar se não souber o que está fazendo:
      unicode=true,
      pdflang={pt-BR},
      bookmarksopen=true,
      bookmarksnumbered=true,
      bookmarksopenlevel=5,
      pdfdisplaydoctitle=true,
      pdfpagemode=UseOutlines,
      pdfstartview=FitH,
      pdfcreator={LaTeX with hyperref package},
      pdfproducer={pdfTeX},
      pdfnewwindow=true,
      colorlinks=true,
      citecolor=green,
      linkcolor=red,
      filecolor=cyan,
      urlcolor=blue
    }
\else
  \usepackage{hyperref}
\fi
\usepackage{cleveref}
\usepackage{url}


%%%%%%%%%%%%%%%%%%%%%%%%%%%%%%%%%%%%%%%%%%%%%%%%%%%%%%%%%%%%%%%%%%%%%%%%%%%%%%%%%
%%% Referências bibliográficas
\usepackage[round, semicolon, authoryear]{natbib}
\bibliographystyle{abrantesasf}


%%%%%%%%%%%%%%%%%%%%%%%%%%%%%%%%%%%%%%%%%%%%%%%%%%%%%%%%%%%%%%%%%%%%%%%%%%%%%%%%%
%%% Carrega bibliotecas de símbolos (matemáticos, físicos, etc.), fontes
%%% adicionais, e configura algumas opções
\usepackage{amsmath}
\usepackage{amssymb}
\usepackage{amsfonts}
\usepackage{siunitx}
  \sisetup{group-separator = {.}}
  \sisetup{group-digits = {false}}
  \sisetup{output-decimal-marker = {,}}

% Altera separador decimal via comando, se necessário (prefira o siunitx):
%\mathchardef\period=\mathcode`.
%\DeclareMathSymbol{.}{\mathord}{letters}{"3B}
  

%%%%%%%%%%%%%%%%%%%%%%%%%%%%%%%%%%%%%%%%%%%%%%%%%%%%%%%%%%%%%%%%%%%%%%%%%%%%%%%%%
%%% Carrega packages relacionados à computação
\usepackage{algorithm2e}
\usepackage{algorithmicx}
\usepackage{algpseudocode}
\usepackage{listings}
  \lstset{literate=
    {á}{{\'a}}1 {é}{{\'e}}1 {í}{{\'i}}1 {ó}{{\'o}}1 {ú}{{\'u}}1
    {Á}{{\'A}}1 {É}{{\'E}}1 {Í}{{\'I}}1 {Ó}{{\'O}}1 {Ú}{{\'U}}1
    {à}{{\`a}}1 {è}{{\`e}}1 {ì}{{\`i}}1 {ò}{{\`o}}1 {ù}{{\`u}}1
    {À}{{\`A}}1 {È}{{\'E}}1 {Ì}{{\`I}}1 {Ò}{{\`O}}1 {Ù}{{\`U}}1
    {ä}{{\"a}}1 {ë}{{\"e}}1 {ï}{{\"i}}1 {ö}{{\"o}}1 {ü}{{\"u}}1
    {Ä}{{\"A}}1 {Ë}{{\"E}}1 {Ï}{{\"I}}1 {Ö}{{\"O}}1 {Ü}{{\"U}}1
    {â}{{\^a}}1 {ê}{{\^e}}1 {î}{{\^i}}1 {ô}{{\^o}}1 {û}{{\^u}}1
    {Â}{{\^A}}1 {Ê}{{\^E}}1 {Î}{{\^I}}1 {Ô}{{\^O}}1 {Û}{{\^U}}1
    {œ}{{\oe}}1 {Œ}{{\OE}}1 {æ}{{\ae}}1 {Æ}{{\AE}}1 {ß}{{\ss}}1
    {ű}{{\H{u}}}1 {Ű}{{\H{U}}}1 {ő}{{\H{o}}}1 {Ő}{{\H{O}}}1
    {ç}{{\c c}}1 {Ç}{{\c C}}1 {ø}{{\o}}1 {å}{{\r a}}1 {Å}{{\r A}}1
    {€}{{\euro}}1 {£}{{\pounds}}1 {«}{{\guillemotleft}}1
    {»}{{\guillemotright}}1 {ñ}{{\~n}}1 {Ñ}{{\~N}}1 {¿}{{?`}}1
  }
  

%%%%%%%%%%%%%%%%%%%%%%%%%%%%%%%%%%%%%%%%%%%%%%%%%%%%%%%%%%%%%%%%%%%%%%%%%%%%%%%%%
%%% Ativa suporte extendido a cores
\usepackage[svgnames]{xcolor} % Opções de cores: usenames (16), dvipsnames (64),
                              % svgnames (150) e x11names (300).


%%%%%%%%%%%%%%%%%%%%%%%%%%%%%%%%%%%%%%%%%%%%%%%%%%%%%%%%%%%%%%%%%%%%%%%%%%%%%%%%%
%%% Suporte à importação de gráficos externos
\ifpdf
  \usepackage[pdftex]{graphicx}
\else
  \usepackage[dvips]{graphicx}
\fi


%%%%%%%%%%%%%%%%%%%%%%%%%%%%%%%%%%%%%%%%%%%%%%%%%%%%%%%%%%%%%%%%%%%%%%%%%%%%%%%%%
%%% Suporte à criação de gráficos proceduralmente na LaTeX:
\usepackage{tikz}
  \usetikzlibrary{arrows,automata,backgrounds,matrix,patterns,positioning,shapes,shadows}


%%%%%%%%%%%%%%%%%%%%%%%%%%%%%%%%%%%%%%%%%%%%%%%%%%%%%%%%%%%%%%%%%%%%%%%%%%%%%%%%%
%%% Packages para tabelas
\usepackage{array}
\usepackage{longtable}
\usepackage{tabularx}
\usepackage{tabu}
\usepackage{lscape}
\usepackage{colortbl}  
\usepackage{booktabs}


%%%%%%%%%%%%%%%%%%%%%%%%%%%%%%%%%%%%%%%%%%%%%%%%%%%%%%%%%%%%%%%%%%%%%%%%%%%%%%%%%
%%% Packages ambientes de listas
\usepackage{enumitem}
\usepackage[ampersand]{easylist}


%%%%%%%%%%%%%%%%%%%%%%%%%%%%%%%%%%%%%%%%%%%%%%%%%%%%%%%%%%%%%%%%%%%%%%%%%%%%%%%%%
%%% Packages para suporte a ambientes floats, captions, etc.:
\usepackage{float}
\usepackage{wrapfig}
\usepackage{placeins}
\usepackage{caption}
\usepackage{sidecap}
\usepackage{subcaption}


%%%%%%%%%%%%%%%%%%%%%%%%%%%%%%%%%%%%%%%%%%%%%%%%%%%%%%%%%%%%%%%%%%%%%%%%%%%%%%%%%
%%% Meus comandos específicos:
% Commando para ``italizar´´ palavras em inglês (e outras línguas!):
\newcommand{\ingles}[1]{\textit{#1}}

% Commando para colocar o espaço correto entre um número e sua unidade:
\newcommand{\unidade}[2]{\ensuremath{#1\,\mathrm{#2}}}
\newcommand{\unidado}[2]{{#1}\,{#2}}

% Produz ordinal masculino ou feminino dependendo do segundo argumento:
\newcommand{\ordinal}[2]{%
#1%
\ifthenelse{\equal{a}{#2}}%
{\textordfeminine}%
{\textordmasculine}}


%%%%%%%%%%%%%%%%%%%%%%%%%%%%%%%%%%%%%%%%%%%%%%%%%%%%%%%%%%%%%%%%%%%%%%%%%%%%%%%%%
%%% Hifenização específica quando o LaTeX/Babel não conseguirem hifenizar:
\babelhyphenation{Git-Hub}


%%%%%%%%%%%%%%%%%%%%%%%%%%%%%%%%%%%%%%%%%%%%%%%%%%%%%%%%%%%%%%%%%%%%%%%%%%%%%%%%%
%%%%%%%%%%%%%%%%%%%%%%%%%%%%%%%%%%%%%%%%%%%%%%%%%%%%%%%%%%%%%%%%%%%%%%%%%%%%%%%%%
%%%%%%%%%%%%%%%%%%%%%%%%%%%%%%%%%%%%%%%%%%%%%%%%%%%%%%%%%%%%%%%%%%%%%%%%%%%%%%%%%
%%%%%%%%%%%%%%%%%%%%%%%%%%%%%%%%%%%%%%%%%%%%%%%%%%%%%%%%%%%%%%%%%%%%%%%%%%%%%%%%%
%%%%%%%%%%%%%%%%%%%%%%%%%%%%%% COMEÇA O DOCUMENTO %%%%%%%%%%%%%%%%%%%%%%%%%%%%%%%
%%%%%%%%%%%%%%%%%%%%%%%%%%%%%%%%%%%%%%%%%%%%%%%%%%%%%%%%%%%%%%%%%%%%%%%%%%%%%%%%%
%%%%%%%%%%%%%%%%%%%%%%%%%%%%%%%%%%%%%%%%%%%%%%%%%%%%%%%%%%%%%%%%%%%%%%%%%%%%%%%%%
%%%%%%%%%%%%%%%%%%%%%%%%%%%%%%%%%%%%%%%%%%%%%%%%%%%%%%%%%%%%%%%%%%%%%%%%%%%%%%%%%
%%%%%%%%%%%%%%%%%%%%%%%%%%%%%%%%%%%%%%%%%%%%%%%%%%%%%%%%%%%%%%%%%%%%%%%%%%%%%%%%%
\begin{document}
\title{Internet das Coisas}
\author{Abrantes Araújo Silva Filho \and Brendhom Félix Garcia Brito
  \and Carlos Augusto Caus Couto \and Eliziel de Paula da Silva \and Iuri Alves Contarelli
  \and Victor Luchsinger Lube}
\date{2018-04-19}
\maketitle
\tableofcontents


%%%%%%%%%%%%%%%%%%%%%%%%%%%%%%%%%%%%%%%%%%%%%%%%%%%%%%%%%%%%%%%%%%%%%%%%%%%%%%%%%
%%%%%%%%%%%%%%%%%%%%%%%%%%%%%%%%%%%%%%%%%%%%%%%%%%%%%%%%%%%%%%%%%%%%%%%%%%%%%%%%%
%%%%%%%%%%%%%%%%%%%%%%%%%%%%%%%%%%%%%%%%%%%%%%%%%%%%%%%%%%%%%%%%%%%%%%%%%%%%%%%%%
\section{Introdução}
\label{intro}

Nos últimos anos a tecnologia permitiu a criação e o desenvolvimento de dispositivos com a
capacidade de se conectar à Internet e de trocar informações. Esses dispositivos,
segundo \citet{BarbozaTCCIoT2015}, ``são objetos físicos, 'coisas', que passam
a alojar sistemas eletrônicos embarcados com componentes de \ingles{software},
sensores e, principalmente conectividade, que permite a esses objetos trocarem
informações através da rede''.

Apesar de não haver um consenso na definição do campo de ``dispositivos conectados'',
o nome Internet das Coisas --- do inglês \ingles{Internet of Things} (IoT)\footnote{Neste trabalho
  usaremos o termo ``IoT'' para indicar todo o ecossistema de dispositivos conectados.} --- passou
a significar todo o ecossistema de \ingles{hardware} e \ingles{software} que
permite a conexão dos dispositivos entre si e à Internet.

A produção em escala desses dispositivos para IoT com a conseqüente queda no custo
de produção, aliado à enorme distribuição de redes de conectividade \ingles{wireless},
expandiu de forma exponencial as possibilidades de uso e lucratividade com
tal tecnologia. No ano de 2013 a Cisco estimou que esse potencial
de mercado era de 14,4 trilhões de dólares em dez anos\footnote{De 2013 a 2022, considerando
o mercado global.}~\citep{CiscoIoTFAQ2013},
e que até 2020 haverá 50,1 bilhões de dispositivos conectados à Internet,
realizando diversas tarefas e serviços~\citep{CiscoIoEConnectionsCounter}.

Além das empresas de tecnologia, grandes consultoras de negócios internacionais
já apontam o enorme potencial da IoT, explicitamente aconselhando
seus clientes a investirem na área.
A Morgan Stanley publicou dois relatórios~\citep{MorganStanleyIoTnow2013,MorganStanleyIoTpersonal2014}
que apontam enorme potencial de ganhos, e a \citet{OliverWymanIoT2015} classifica
a IoT como ``uma nova revolução capaz de romper os modelos de negócios tradicionais''.

Até mesmo governos, que geralmente são mais lentos na adoção de novas tecnologias, já
se movimentam para estudar, testar e implantar soluções concetadas. O Reino Unido,
por exemplo, estabeleceu para si a meta de se tornar, através do investimento em IoT,
a ``nação mais digital entre os componentes do G8\footnote{França, Alemanha, Reino Unido,
  Japão, Canadá, Itália, Estados Unidos e Rússia (esta foi afastada recentemente sob
  a acusação de violaão da soberania nacional da Ucrânia).}'' e 
``o líder mundial no desenvolvimento e implementação da IoT''~\citep{UKGOSWalportIoT2014}.

E por que toda essa agitação em torno da IoT? Porque as suas aplicações
são praticamente inesgotáveis, variando desde uma simples
\ingles{Smart-TV} capaz de se conectar à Internet e atualizar a programação de
filmes disponíveis ou de um simples sistema de irrigação de jardim capaz de
obter a previsão de chuva a partir de serviços na Internet e ajustar a periodicidade
na qual a grama será molhada, até dispositivos complexos como um marca-passo cardíaco
capaz de informar automaticamente ao médico ou uma equipe de emergência uma possível
disfunção miocárdica que exija tratamento imediato.

Este trabalho pretende apresentar uma visão geral do campo da IoT, incluindo os
seguintes tópicos:

\begin{itemize}[noitemsep]
\item Definições e conceitos
\item Histórico
\item Vantagens e desvantagens
\item Riscos e perigos
\item Exemplos de aplicações (que deram certo e errado)
\item Tecnologias utilizadas
\end{itemize}

Além disso este trabalho apresentará uma breve descrição de um protótipo de
cofre conectado à Internet por um Arduino\footnote{\url{https://www.arduino.cc/}},
que será desenvolvido pelos autores como uma prova de conceito da IoT.


%%%%%%%%%%%%%%%%%%%%%%%%%%%%%%%%%%%%%%%%%%%%%%%%%%%%%%%%%%%%%%%%%%%%%%%%%%%%%%%%%
%%%%%%%%%%%%%%%%%%%%%%%%%%%%%%%%%%%%%%%%%%%%%%%%%%%%%%%%%%%%%%%%%%%%%%%%%%%%%%%%%
\section{Conceitos e definições}
\label{conceitos}


%%%%%%%%%%%%%%%%%%%%%%%%%%%%%%%%%%%%%%%%%%%%%%%%%%%%%%%%%%%%%%%%%%%%%%%%%%%%%%%%%
\subsection{O que é IoT?}
\label{conceitos-o-que-e-iot}

Não existe um consenso estabelecido sobre o que realmente é a IoT e qual a melhor
maneira de definí-la. Isso ocorre devido a relativa pouca idade e maturação do
campo, devido às diverentes visões dos dispositivos pioneiros de IoT, e devido
ao praticamente ilimitado potencial de uso para diferentes atividades e serviços
em diversas áreas (indústria, doméstica, saúde, financeira, engenharia e muitas
outras). Se a IoT estará presente em ``tudo'' e servirá para ``tudo'', como
definí-la precisamente?

Inicialmente vamos eliminar uma visão simplória e quase caricata da IoT: a
``geladeira conectada que compra leite fresco'' (Figura~\ref{fig:smart-geladeira}).
\citet{UKGOSWalportIoT2014}
argumenta que esse tipo de estereótipo somente contribui para ``trivializar
a importância da IoT'' e mascarar o verdadeiro potencial de impacto na sociedade.
Eletrodomésticos conectados são uma pequena parte da IoT, mas não a mais
importante.

\begin{figure}[H]
  \begin{center}
    \caption{``A porta-voz da Samsung, Kai Madden, exibe o recurso de conectividade em
      uma geladeira inteligente Samsung''~\citep{BajarinIoE2014}.}
    \label{fig:smart-geladeira}
    \fbox{\includegraphics[scale=0.8]{imagens/geladeira.jpeg}}
    \newline
    \footnotesize{Foto de David Becker, retirada do artigo de ~\citet{BajarinIoE2014},
      disponível em \url{http://time.com/539/the-next-big-thing-for-tech-the-internet-of-everything/}}
  \end{center}
\end{figure}

Se a IoT não se resume a eletrodomésticos conectados, o que ela é de fato? As
grandes empresas de tecnologia definem IoT do seguinte modo:

\begin{itemize}
\item \emph{\textbf{SAP}}: ``A Internet das Coisas é uma rede de objetos físicos --- veículos,
  máquinas, eletrodomésticos e outros --- que usam sensores e APIs para conectar e
  trocar dados na Internet''~\citep{SAPWhatIoT}.
\item \emph{\textbf{SAS}}: ``A Internet das Coisas é o conceito de objetos do cotidiano --- de
  máquinas industriais à dispositivos vestíveis ---- usando sensores embutidos para coletar
  dados e tomar uma ação sobre esses dados através da rede''\citep{SASWhatIoT}.
\item \emph{\textbf{IBM}}: ``A Internet das Coisas refere-se à variedade crescente
  de dispositivos conectados que enviam dados através da Internet. Uma 'coisa' é qualquer
  objeto com eletrônica embarcada que pode transferir dados em uma rede --- sem nenhuma
  interação humana''~\citep{IBMWhatIsIoT,IBMWhatsonIoT}.
\item \emph{\textbf{Cisco}}: ``A Internet das Coisas (IoT) refere-se simplesmente à conexão
  em rede de objetos físicos''~\citep{CiscoIoTVS2013}.
\item \emph{\textbf{Amazon}}: ``Um sistema de dispositivos ubíquos conectando o mundo
  físico à nuvem''~\citep{AmazonIoT}.
\item \emph{\textbf{Microsoft}}: embora a Microsoft não defina explicitamente o que
  ela entende por IoT, em um vídeo institucional sobre a plataforma \emph{Microsoft IoT}
  podemos deduzir que o conceito de IoT envolve conectar as partes mais vitais de
  seu negócio (pessoas, ativos, processos e sistemas), do chão de fábrica aos ``campos''
  para aumentar o alcançe da empresa e fazer melhor uso dos recursos~\citep{MicrosoftIoT}.
\item \emph{\textbf{Google}}: a empresa também não fornece uma definição explícita mas
  sua plataforma \emph{Google Cloud IoT Core} está voltada à ``conexão segura, coleta e
  gerenciamento de dados a partir de milhões de dispositivos globalmente dispersos [\ldots]
  para processar, analisar e visualizar dados em tempo real para suportar o aumento
  da eficiência operacional''~\citep{GoogleWhatsIoT}.
\end{itemize}



%%%%%%%%%%%%%%%%%%%%%%%%%%%%%%%%%%%%%%%%%%%%%%%%%%%%%%%%%%%%%%%%%%%%%%%%%%%%%%%%%
%%%%%%%%%%%%%%%%%%%%%%%%%%%%%%%%%%%%%%%%%%%%%%%%%%%%%%%%%%%%%%%%%%%%%%%%%%%%%%%%%
%%%%%%%%%%%%%%%%%%%%%%%%%%%%%%%%%%%%%%%%%%%%%%%%%%%%%%%%%%%%%%%%%%%%%%%%%%%%%%%%%
%%%%%%%%%%%%%%%%%%%%%%%%%%%%%%%%%%%%%%%%%%%%%%%%%%%%%%%%%%%%%%%%%%%%%%%%%%%%%%%%%
%%%%%%%%%%%%%%%%%%%%%%%%%%%%%% TERMINA O DOCUMENTO %%%%%%%%%%%%%%%%%%%%%%%%%%%%%%
%%%%%%%%%%%%%%%%%%%%%%%%%%%%%%%%%%%%%%%%%%%%%%%%%%%%%%%%%%%%%%%%%%%%%%%%%%%%%%%%%
%%%%%%%%%%%%%%%%%%%%%%%%%%%%%%%%%%%%%%%%%%%%%%%%%%%%%%%%%%%%%%%%%%%%%%%%%%%%%%%%%
%%%%%%%%%%%%%%%%%%%%%%%%%%%%%%%%%%%%%%%%%%%%%%%%%%%%%%%%%%%%%%%%%%%%%%%%%%%%%%%%%
%%%%%%%%%%%%%%%%%%%%%%%%%%%%%%%%%%%%%%%%%%%%%%%%%%%%%%%%%%%%%%%%%%%%%%%%%%%%%%%%%
\bibliography{/home/abrantesasf/repositoriosGit/BibTeX/biblioteca}
\end{document}
